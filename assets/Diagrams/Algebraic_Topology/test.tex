\documentclass[12pt]{article}
\usepackage{amsmath, amssymb, amsthm}

\title{The Algebraic Proof of the Universal Coefficient Theorem for Homology}
\author{}
\date{}

\begin{document}

\maketitle

\section*{1. The Setup}

Let \( C_\ast \) be a chain complex of free abelian groups with boundary maps \( \partial_n \).
\begin{itemize}
    \item Define \( Z_n = \ker(\partial_n) \) (cycles) and \( B_n = \operatorname{im}(\partial_{n+1}) \) (boundaries).
    \item The homology is \( H_n(C) = Z_n / B_n \).
\end{itemize}

\section*{2. Tensoring with a Coefficient Group}

Given any abelian group \( G \), tensor each group with \( G \):
\begin{itemize}
    \item Tensoring is right exact, so exactness may be lost on the left.
\end{itemize}

\section*{3. Short Exact Sequences at Each Degree}

At each degree \( n \), we have:
\[
0 \longrightarrow Z_n \longrightarrow C_n \xrightarrow{\partial_n} B_{n-1} \longrightarrow 0
\]
Tensoring with \( G \) (preserves exactness for free groups):
\[
0 \longrightarrow Z_n \otimes G \longrightarrow C_n \otimes G \xrightarrow{\partial_n \otimes 1} B_{n-1} \otimes G \longrightarrow 0
\]

\section*{4. Diagram for the Snake Lemma}

Aligning the sequences for degrees \( n \) and \( n-1 \), with vertical maps given by boundaries, you get:

\[
\begin{array}{ccc}
Z_n \otimes G &\longrightarrow& C_n \otimes G \longrightarrow B_{n-1} \otimes G \\
\downarrow 0 && \downarrow \partial_n \otimes 1 \quad \downarrow 0 \\
Z_{n-1} \otimes G &\longrightarrow& C_{n-1} \otimes G \longrightarrow B_{n-2} \otimes G
\end{array}
\]

\section*{5. Applying the Snake Lemma}

The snake lemma gives the long exact sequence:
\[
0 \rightarrow Z_n \otimes G \rightarrow \ker(\partial_n \otimes 1) \rightarrow B_{n-1} \otimes G \xrightarrow{\delta} Z_{n-1} \otimes G \rightarrow \cdots
\]
where \( \ker(\partial_n \otimes 1) \) are the cycles in \( C_n \otimes G \).

The homology with coefficients in \( G \) is
\[
H_n(C \otimes G) = \frac{\ker(\partial_n \otimes 1)}{\operatorname{im}(\partial_{n+1} \otimes 1)}
\]

\section*{6. From Cycles to Homology and the Role of Tor}

Naively, one expects all homology with coefficients to come from \( H_n(C) \otimes G = (Z_n / B_n) \otimes G \).
However, boundaries may ``expand'' or interact nontrivially with \( G \), so new elements can appear in \( H_n(C \otimes G) \).
The ``error term'' is measured by
\[
\operatorname{Tor}_1(H_{n-1}(C), G)
\]
which captures classes from the torsion in \( H_{n-1}(C) \) and its interaction with \( G \).

\section*{7. The Universal Coefficient Theorem for Homology}

The final statement is the short exact sequence:
\[
0 \longrightarrow H_n(C) \otimes G \longrightarrow H_n(C \otimes G) \longrightarrow \operatorname{Tor}_1(H_{n-1}(C), G) \longrightarrow 0
\]
where
\begin{itemize}
    \item \( H_n(C) \otimes G \): contributions from (integral) homology tensored with \( G \),
    \item \( H_n(C \otimes G) \): full homology with coefficients in \( G \),
    \item \( \operatorname{Tor}_1(H_{n-1}(C), G) \): the error term from torsion (failure of left exactness).
\end{itemize}

\section*{8. Conclusion}
This algebraic proof uses the properties of chain complexes, tensor products, and the snake lemma to explain the origin and meaning of all terms in the universal coefficient theorem for homology.


\section*{How the Tor Term Appears from $B_{n-1}\otimes G$}

Consider the short exact sequence
\[
0 \to B_{n-1} \to Z_{n-1} \to H_{n-1}(C) \to 0
\]
where $B_{n-1}$ is the group of boundaries, $Z_{n-1}$ the cycles, and $H_{n-1}(C)$ the homology group.

Tensor with $G$:
\[
B_{n-1} \otimes G \xrightarrow{\alpha} Z_{n-1} \otimes G \xrightarrow{\beta} H_{n-1}(C) \otimes G \to 0
\]
This sequence is always right exact, but may fail to be left exact.

The failure of exactness on the left is captured by $\operatorname{Tor}_1$:
\[
0 \to \operatorname{Tor}_1(H_{n-1}(C), G) \to B_{n-1} \otimes G \xrightarrow{\alpha} Z_{n-1} \otimes G \xrightarrow{\beta} H_{n-1}(C) \otimes G \to 0
\]
Here, $\operatorname{Tor}_1(H_{n-1}(C), G)$ is defined as the kernel of $\alpha$.

This is precisely the additional "torsion" contribution that is part of the Universal Coefficient Theorem:
\[
0 \longrightarrow H_n(C) \otimes G \longrightarrow H_n(C \otimes G) \longrightarrow \operatorname{Tor}_1(H_{n-1}(C), G) \longrightarrow 0
\]

\paragraph{Example:}
If $H_{n-1}(C) = \mathbb{Z}/n$ and $G = \mathbb{Z}/m$, then
\[
\operatorname{Tor}_1(\mathbb{Z}/n, \mathbb{Z}/m) \cong \mathbb{Z}/\gcd(n, m)
\]
demonstrating explicitly how elements in $B_{n-1} \otimes G$ may be "killed" under the map to $Z_{n-1} \otimes G$, revealing torsion in the homology with coefficients.

\newpage

\section*{Universal Coefficient Theorem for Cohomology via Homology UCT and Hom-Tensor Adjunction}

Let $C_\ast$ be a chain complex of free abelian groups (as occurs for the standard chain complexes in algebraic topology), and let $G$ be any abelian group.

\subsection*{1. Universal Coefficient Theorem for Homology}

For each $n$, the UCT for homology gives a short exact sequence
\[
0 \to H_n(C_\ast) \otimes G \xrightarrow{\alpha} H_n(C_\ast; G) \xrightarrow{\beta} \operatorname{Tor}_1(H_{n-1}(C_\ast), G) \to 0
\]
where $H_n(C_\ast) = Z_n/B_n$ is the $n$-th homology group, and $H_n(C_\ast; G) := H_n(C_\ast \otimes G)$.

\subsection*{2. Cohomology and Hom-Tensor Adjunction}

Define the cochain complex $C^\ast := \operatorname{Hom}(C_\ast, G)$, with
\[
H^n(C_\ast; G) := H^n(\operatorname{Hom}(C_\ast, G)).
\]
The functor $\operatorname{Hom}(-, G)$ is left exact, with right derived functors $\operatorname{Ext}^k(-, G)$.

The Hom-Tensor adjunction tells us that
\[
\operatorname{Hom}(A \otimes G, K) \cong \operatorname{Hom}(A, \operatorname{Hom}(G, K))
\]
for abelian groups $A, G, K$.

\subsection*{3. Deduction of the Cohomology UCT}

Consider the short exact sequence from the UCT for homology:
\[
0 \longrightarrow H_n(C_\ast) \otimes G \longrightarrow H_n(C_\ast; G) \longrightarrow \operatorname{Tor}_1(H_{n-1}(C_\ast), G) \longrightarrow 0.
\]
Apply the contravariant functor $\operatorname{Hom}(-, G)$ and use the fact that $\operatorname{Ext}^1(-, G)$ is its right derived functor. From homological algebra, this gives a long exact sequence:
\[
0 \to \operatorname{Hom}(H_{n-1}(C_\ast), G) \to \operatorname{Hom}(Z_{n-1}, G) \to \operatorname{Hom}(B_{n-1}, G)
\to \operatorname{Ext}^1(H_{n-1}(C_\ast), G) \to \cdots
\]

By analyzing the chain maps and the computation of $H^n(\operatorname{Hom}(C_\ast,G))$, one obtains the exact sequence:
\[
0 \longrightarrow \operatorname{Ext}^1(H_{n-1}(C_\ast), G) \longrightarrow H^n(C_\ast; G) \longrightarrow \operatorname{Hom}(H_n(C_\ast), G) \longrightarrow 0.
\]

\subsection*{4. Conclusion}

By properties of derived functors, the left exactness of $\operatorname{Hom}(-, G)$ and the exactness of $C_\ast$, the UCT for cohomology is a formal consequence of the UCT for homology and the Hom-Tensor adjunction.

\bigskip
\noindent \textbf{Summary Table:}
\[
\begin{array}{c|c}
\text{UCT for Homology} & \text{UCT for Cohomology} \\
\hline
0 \to H_n(C_\ast) \otimes G \to H_n(C_\ast;G) \to \operatorname{Tor}_1(H_{n-1}(C_\ast), G) \to 0 & 0 \to \operatorname{Ext}^1(H_{n-1}(C_\ast), G) \to H^n(C_\ast;G) \to \operatorname{Hom}(H_n(C_\ast), G) \to 0
\end{array}
\]

\end{document}

